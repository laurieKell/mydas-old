% $Id: $
\documentclass[a4paper, 10pt]{article}
% reduced margins
\usepackage{fullpage}
\usepackage[authoryear,round]{natbib}
% spacing		
\usepackage{setspace}
% page headings
\usepackage{fancyhdr}
%\usepackage{lscape}

\let\subequations\relax

\usepackage{pdflscape}
\usepackage{pdfpages}
\usepackage{grffile}


\usepackage[acronym,toc]{glossaries} % nomain, if you define glossaries in a file, and you use \include{INP-00-glossary}
%\input{glossary.tex}
%\makeglossaries

\usepackage{subfigure} 
\usepackage[margin=1.0in]{geometry}
\usepackage{url}
%\usepackage{subeqn}
\usepackage{multirow}
\usepackage{booktabs}

\usepackage{enumerate}
\usepackage{enumitem}

\setlength{\headheight}{15.2pt}
\pagestyle{fancy}

\usepackage{graphicx}
\usepackage{color}
\usepackage{hyperref}
\usepackage{url}
\hypersetup{colorlinks, urlcolor=darkblue}

\providecommand{\tightlist}{%
  \setlength{\itemsep}{0pt}\setlength{\parskip}{0pt}}


\usepackage{lscape}
% figs to be 75% of test width
\setkeys{Gin}{width=0.75\textwidth}


\newcommand{\mathgloss}[2]{
 \newglossaryentry{#1}{name={#1},description={#2}}
 \gls{#1} = #2
}
\newcommand{\ds}{\displaystyle}
\newcommand{\eps}{\epsilon}
\newcommand{\veps}{\varepsilon}
\newcommand{\wh}{\widehat}

%
\renewcommand{\abstractname}{\large SUMMARY}
%
\newcommand{\Keywords}[1]{\begin{center}\par\noindent{{\em KEYWORDS\/}: #1}\end{center}}
%
\makeatletter
\renewcommand{\subsubsection}{\@startsection{subsubsection}{3}{\z@}%
 {-1.25ex\@plus -1ex \@minus -.2ex}%
 {1.5ex \@plus .2ex}%
 {\normalfont\slshape}}
\renewcommand{\subsection}{\@startsection{subsection}{2}{\z@}%
 {-3.25ex\@plus -1ex \@minus -.2ex}%
 {1.5ex \@plus .2ex}%
 {\normalfont\bfseries\slshape}}
\renewcommand{\section}{\@startsection{section}{1}{\z@}%
 {-5.25ex\@plus -1ex \@minus -.2ex}%
 {1.5ex \@plus .2ex}%
 {\normalfont\bfseries}}
\makeatother
%
\renewcommand\thesection{\arabic{section}.}
\renewcommand\thesubsection{\thesection\arabic{subsection}}
\renewcommand\thesubsubsection{\thesubsection\arabic{subsubsection}}
%
\renewcommand{\headrulewidth}{0pt}

\usepackage{listings}

\newenvironment{mylisting}
{\begin{list}{}{\setlength{\leftmargin}{1em}}\item\scriptsize\bfseries}
{\end{list}}

\newenvironment{mytinylisting}
{\begin{list}{}{\setlength{\leftmargin}{1em}}\item\tiny\bfseries}
{\end{list}}

\usepackage{listings}

\definecolor{darkblue}{rgb}{0,0,0.5}
\definecolor{shadecolor}{rgb}{1,1,0.95}
\definecolor{shade}{rgb}{1,1,0.95}


\lstset{ %
language=R, % the language of the code
basicstyle=\footnotesize, % the size of the fonts that are used for the code
numbers=left, % where to put the line-numbers
numberstyle=\footnotesize, % the size of the fonts that are used for the line-numbers
stepnumber=1	00, % the step between two line-numbers. If it's 1, each line 
 % will be numbered
numbersep=5pt, % how far the line-numbers are from the code
backgroundcolor=\color{shade}, % choose the background color. You must add \usepackage{color}
showspaces=false, % show spaces adding particular underscores
showstringspaces=false, % underline spaces within strings
showtabs=false, % show tabs within strings adding particular underscores
frame=single, % adds a frame around the code
tabsize=2, % sets default tabsize to 2 spaces
captionpos=b, % sets the caption-position to bottom
breaklines=true, % sets automatic line breaking
breakatwhitespace=false, % sets if automatic breaks should only happen at whitespace
title=\lstname, % show the filename of files included with \lstinputlisting;
 % also try caption instead of title
escapeinside={\%*}{*)}, % if you want to add a comment within your code
morekeywords={*,...} presents % if you want to add more keywords to the set
}

%
\title{Evaluation ofICES \textit{2 over 3} Rule}
%
\author{Laurence T. Kell\footnote{Imperial College London; ~laurie@seaplusplus.co.uk.},
        Alexander Tidd\footnote{GMIT;},
        Coilin Minto\footnote{GMIT;}}
\date{\today}
%


\begin{document}

\onehalfspacing
\lhead{\normalsize\textsf{Workshop on the Development of Quantitative Assessment Methodologies based on Life-history traits for stocks in categories 3-6 (WKLIFE VIII)}}
\rhead{}

\maketitle
% gets headers on title page ...
\thispagestyle{fancy}
% ... but not on others
\pagestyle{empty}


\tableofcontents\newpage\
%\listoffigures\newpage


\section*{Outline}

\begin{itemize}

 \item ICES uses a rule of the form $C_{y+1} = C_{current}rfb$ for category 3 stocks. Decompose this rule into its elements, and then look at their relative importance and interactions. 
 \item This can be done by first by performing some simulations, i.e. use the OM/OEM without feedback to test candidate methods. If a method can not get a reasonable estimate of the stock status why use it in an MP?
\item rfb
\begin{description}
 \item[Ct:] How much to change the TAC by, a lot for a short lived species with a high r and variable recruitment like sprat or only a little bit for Ray? 
 \item[r:] (average of stock size index in the 2 most recent years)/(average of stock indexin the predeeding three years). Why 2 over 3? does it depend on life history?, why not dI/dt, i.e. slope of Index.
 \item[f:] Length bases proxies for f/fmsy. i.e. based on length. LBSPR seems to work, but Gedamke/hoenig is not looking good      
  \item[b:] This is a the HCR: i.e. with biomass targets and limits. These could be reference points or reference periods.
\end{description}
\item Once we have a better understanding of the elements we can then come up with the ones that are worth pursuing and tune some MPs. Comparing across life histories will make this work particularly useful.
\item We can also use machine learning to come up with "better" MPs.
\end{itemize}



\section{Introduction}

In this study we use Management Strategy Evaluation (MSE) to test advice rules that utilise length-based approaches and advice rules. 

ICES uses a rule of the form $C_{y+1} = C_{current}rfb$.
 
The specific aims of the study are to

\begin{itemize}
 \item Establish whether performance of the advice rules is correlated with life-history characteristics
 \item If such correlations exist, develop guidelines for use of the advice rules dependent on life-history characteristics
\end{itemize}
 
\section{Material and Methods}

\begin{itemize}
 \item ~
 \item ~
\end{itemize}

\section{Results}

\begin{itemize}
 \item ~
 \item ~
\end{itemize}

\section{Discussion and Conclusions}

\begin{itemize}
 \item ~
 \item ~
\end{itemize}

\section{Acknowledgements}

This paper was written under the MyDas project funded by the Irish exchequer and EMFF 2014-2020. The overall aim of MyDas is to develop and test a range of assessment models and methods to establish Maximum Sustainable Yield (MSY) reference points (or proxy MSY reference points) across the spectrum of data-limited stocks.


%\newpage\clearpage
\bibliography{refs.bib} 
\bibliographystyle{abbrvnat}

\clearpage\newpage 
\section{Figures}

\end{document}

\begin{figure}[htbp]\centering\includegraphics[width=\textwidth]{}\label{}\end{figure}
\clearpage\newpage 
\section{Appendices}
\newpage 

\subsection{Bibliography of materials provided for review} 
\includepdf[pages=-,width=\textwidth]{/home/laurence/Desktop/sea++/cie/greySnapper/docs/S51DocList.pdf}

\subsection{Statement of Work} 
\includepdf[pages=2-6,width=\textwidth]{/home/laurence/Desktop/sea++/cie/greySnapper/docs/KellPOIEPR-17-052.pdf}

\subsection{Panel membership.} 
\includepdf[width=\textwidth]{/home/laurence/Desktop/sea++/cie/greySnapper/tex/participants.pdf}

\subsection{Additional Analyses.} 

\newpage\clearpage\subsubsection{Indices of Abundance}
\includepdf[pages=-,width=\textwidth]{/home/laurence/Desktop/sea++/cie/greySnapper/R/a1-cpue.pdf}
\subsubsection{Residuals from Fits to Surveys}
\includepdf[pages=-,width=\textwidth]{/home/laurence/Desktop/sea++/cie/greySnapper/R/a2-cpue-residuals.pdf}
\subsubsection{ALK and Cohort Strengths}
\includepdf[pages=-,width=\textwidth]{/home/laurence/Desktop/sea++/cie/greySnapper/R/a3-alk.pdf}
\subsubsection{Fits to length composition data}
\includepdf[pages=-,width=\textwidth]{/home/laurence/Desktop/sea++/cie/greySnapper/R/a4-lenComps.pdf}
\subsubsection{Stock and recruitment}
\includepdf[pages=-,width=\textwidth]{/home/laurence/Desktop/sea++/cie/greySnapper/R/a5-srr.pdf}

\end{document}

